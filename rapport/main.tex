\documentclass[10pt]{article}
\usepackage{Preamble}

\begin{document}

\begin{titlepage}
    \newgeometry{margin=3cm}
    \centering
    \includegraphics[width=0.4\linewidth]{epfl.png}\\[0.5cm]
    \textsc{\LARGE École Polytechnique Fédérale de Lausanne}\\[1cm]
    \textsc{\Large CS-433 Machine Learning}\\[0.3cm]
    \textsc{\large Project 2 -- Fall 2025}\\
    \vspace{\fill}
    \rule{\linewidth}{0.5mm}\\[0.5cm]
    {\huge\bfseries Date Prediction for Museum Collections\\
    \vspace{0.3cm}Using BERT Embeddings}\\[0.3cm]
    \rule{\linewidth}{0.5mm}\\
    \vspace{\fill}
    \textsc{\large In collaboration with}\\[0.3cm]
    \textbf{Laboratory for Experimental Museology (eM+)}\\[1cm]
    \begin{tabular}{l l}
        \textbf{Louis Larcher}      & SCIPER: 361002 \\
        \textbf{Cassio Manueguerra} & SCIPER: 346232 \\
        \textbf{Arthur Taieb}       & SCIPER: 361195 \\
    \end{tabular}\\[2cm]
    \today
\end{titlepage}
\restoregeometry

\thispagestyle{numberonly}
\begin{summary}
    \section*{Abstract}
    This report presents our work on predicting creation dates for artworks in the Wellcome Collection,
    conducted in collaboration with the Laboratory for Experimental Museology (eM+) at EPFL.
    Given the incomplete metadata in museum digital archives, automated date prediction offers
    significant value for collection curation and research. We explore two approaches:
    (1) using frozen BERT embeddings combined with a linear classifier, and
    (2) fine-tuning BERT end-to-end with the classification head. Our methods leverage textual descriptions,
    titles, and other metadata fields to predict temporal categories. We evaluate both approaches and
    discuss their trade-offs in terms of accuracy and computational requirements.
\end{summary}

\section{Introduction}
\subfile{Introduction}

\subfile{Methods}

\subfile{Results}

\section{Conclusion}
\subfile{Conclusion}

\clearpage
\pagestyle{numberonly}
\printbibliography

\end{document}
