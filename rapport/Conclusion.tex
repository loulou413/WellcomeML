% !TEX root = main.tex
\documentclass[main.tex]{subfiles}
\begin{document}

\subsection{Summary}

In this project, we addressed the challenge of automatically predicting creation dates for artworks in the Wellcome Collection, in collaboration with the Laboratory for Experimental Museology (eM+). We developed and compared two approaches based on BERT language models:

\begin{enumerate}
    \item \textbf{Frozen BERT}: Using pre-trained embeddings with a trainable linear classifier, offering computational efficiency while achieving [TO BE FILLED] accuracy.

    \item \textbf{Fine-tuned BERT}: End-to-end training that adapts the model to the date prediction task, achieving [TO BE FILLED] accuracy at higher computational cost.
\end{enumerate}

Our results demonstrate that textual metadata in museum collections contains sufficient signal for temporal prediction, and that modern language models can effectively capture this information.

\subsection{Limitations}

Several limitations should be acknowledged:
\begin{itemize}
    \item The date categories used may not align perfectly with art historical periods
    \item Ambiguous or approximate dates in the ground truth affect model evaluation
    \item Our experiments focused on English-language metadata only
    \item The single-collection focus may limit generalizability
\end{itemize}

\subsection{Future Work}

Promising directions for future research include:
\begin{itemize}
    \item \textbf{Multimodal approaches}: Incorporating image data alongside text for improved predictions
    \item \textbf{Regression formulation}: Predicting continuous year values rather than categorical periods
    \item \textbf{Cross-collection transfer}: Testing generalization to other museum collections
    \item \textbf{Uncertainty quantification}: Providing confidence estimates for predictions
    \item \textbf{Integration with eM+ tools}: Deploying the model in practical museum curation workflows
\end{itemize}

\subsection{Acknowledgments}

We thank the Laboratory for Experimental Museology (eM+) at EPFL for their guidance and domain expertise, and the Wellcome Collection for making their data openly available.

\end{document}
