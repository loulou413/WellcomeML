% !TEX root = main.tex
%!TEX root = main.tex
% !TEX root = ./main.tex
\documentclass[main.tex]{subfiles}
\begin{document}

\subsection{Context and Motivation}

Museum collections worldwide are undergoing large-scale digitization efforts, making cultural heritage
more accessible than ever. The Wellcome Collection \parencite{wellcome2024} is one such
open-access repository containing hundreds of thousands of artworks, manuscripts, and artifacts related
to the history of medicine and human experience.\\
However, like many historical collections, metadata completeness varies significantly-particularly regarding precise dating of objects.
Accurate dating of museum artifacts is crucial for art historical research, exhibition curation,
and understanding the evolution of visual and material culture. Manual annotation by experts is
time-consuming and not scalable to large collections. This motivates the development of automated
methods that can leverage existing textual metadata to predict missing dates.

\subsection{Collaboration with eM+}

This project is conducted in collaboration with the Laboratory for Experimental Museology (eM+) at EPFL. Initial discussions with the lab explored various potential contributions, including:
\begin{itemize}
    \item Visualization of collection embeddings for exploratory analysis
    \item Recommendation systems for similar artworks
\end{itemize}

However, based on feedback from eM+ researchers, we pivoted to focus on \textbf{date prediction}, as this addresses a more immediate and practical need for their work.

\subsection{Problem Statement}

Given textual metadata fields (titles, descriptions, physical description, contributors, etc.)
associated with artworks in the Wellcome Collection, our objective is to predict the
creation date or date range of each item. We formulate this as a classification task,
where the target classes represent temporal periods or date ranges.

\end{document}